\section{Parte 01 - INTRODUCCION BUSINESS MODEL CANVAS} 

\begin{enumerate}[1.]

	\item OBJETIVOS \newline
\\
	\textbf{Objetivo General} \newline
	\subitem * Desarrollar un modelo canvas para un restaurante, que permita contemplar todas las perspectivas organizacionales de la propuesta, con el fin de materializarla y generar valor tanto para sus clientes como para
sus inversionistas. \newline
\\
	\textbf{Objetivos Especificos} \newline
	\subitem * Realizar un diagnóstico del entorno actual en el que se pretende llevar a cabo el plan de negocios.
	\subitem * Plantear el modelo de negocios que permita tener una visión global de la propuesta a través de la metodología Canvas.


	\item MARCO TEORICO \newline
\textbf{¿Qué es un modelo de negocio?} \newline
Un modelo de negocio es una representación que permite entender la manera como una organización crea, entrega y captura valor, y se elabora a partir de preguntas como: ¿Qué es lo que la organización ofrece?, ¿A quién se lo ofrece?, ¿Cómo lo ofrece? Y lo más fundamental, ¿Cómo es qué la organización crea valor a través de su oferta? Londoño
(2008), Osterwalder (2010). El modelo de negocio define las etapas de desarrollo de un proyecto de empresa y es una guía que facilita la creación y el crecimiento de la misma Fleitman (2012). Para efectos de este trabajo, el modelo de negocios seguirá el enfoque del modelo Canvas de Alexander Osterwalder.

	\item INTRODUCCION
	\\\\- El modelo Canvas cuenta con 9 bloques los cuales hacen referencia a las características de la empresa que se quiere crear. Debemos tener en cuenta que al inicio puede costarnos un poco insertar los datos necesarios en cada bloque, y eso puede deberse a que el modelo de negocio aun no está bien definido.\newline
Los 9 bloques para completar son los siguientes:\newline

\textbf {Bloque 1: Segmento de clientes o mercado} \newline
Aquí hablaremos del segmento de personas o entidades que queremos alcanzar. No existe mejor forma de definir quién es tu cliente ideal, que mediante una plantilla de Buyer Persona.\newline
\underline{¿Qué son las Buyer Personas?} \newline
Las buyers personas son representaciones semi-ficticias del cliente ideal de nuestro restaurante. Nos ayudan a definir quién es esta audiencia a la que queremos atraer a nuestro establecimiento y, sobre todo, nos ayuda a humanizar y entender con mayor detalle a esté público objetivo.\newline

\textbf{Bloque 2: Relaciones con los clientes}\newline
En este punto definimos la manera cómo nos vamos a comunicar con nuestro cliente.\newline
¿Dónde queremos que empiece nuestra relación con el cliente?\newline
¿Queremos que acabe en nuestro restaurante?\newline

\textbf{Bloque 3: Canales}\newline
Este punto analiza cómo nuestro restaurante alcanza a nuestro mercado ideal 1 para mostrarles nuestra propuesta de valor 4\newline
Aquí debemos plantearnos la siguiente pregunta: ¿Cómo vamos a lograr que nuestra Buyer Persona acuda a nuestro restaurante?\newline

\textbf{Bloque 4: Propuesta de valor}\newline
Es todo aquello que hace a tu restaurante sea único y diferente al de tu competencia.\newline
Cuando hablo sobre la propuesta de valor no me refiero simplemente al tipo de comida que ofrece un restaurante, sino a todo aquello que nuestro cliente (Buyer persona) valora y por lo que está dispuesto a pagar: trato del personal, diseño del local, tipo de ambiente, localización, etc.\newline
\underline{¿Conoces algún caso de éxito en la que parte de su propuesta de valor no se base solo en su comida?} \newline
Un claro ejemplo es McDonald’s con su Happy Meal.\newline

\textbf{Bloque 5: Fuente de ingresos}\newline
Representa el dinero que genera tu restaurante a través de tu servicio.\newline
Es la consecuencia de todo lo demás.\newline
Normalmente en un restaurante solemos ser muy conservadores y solo se paga por un producto/servicio de forma directa.\newline

\textbf{Bloque 6: Recursos clave}\newline
En este bloque deberemos detectar lo que necesitas para llevar a cabo la actividad de tu restaurante.\newline

\textbf{Bloque 7: Actividades clave}\newline
Identificamos las acciones más importantes que nuestro restaurante debe ejecutar para que nuestro modelo de negocio funcione. \newline

\textbf{Bloque 8: Asociaciones clave}\newline
Debemos especificar cuál es nuestra red de proveedores y socios estratégicos para lograr que nuestro restaurante funcione.\newline
Existe una tendencia cada vez mayor a establecer acuerdos de colaboración con terceros para compartir costes, recursos y experiencias.\newline
Este es un aspecto que se le conoce con el nombre de innovación abierta, ya que la tendencia es a trabajar con más gente y más emprendedores. Se emplea mucho en las Startups de informática.\newline

\textbf{Bloque 9: Estructura de costes}\newline
Analizamos todos los costes necesarios para la viabilidad de nuestro restaurante. \newline

\end{enumerate} 
